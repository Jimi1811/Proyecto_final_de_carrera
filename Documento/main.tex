%% --------------------------------------------------------------------
%% Template UTEC Tesis
%% --------------------------------------------------------------------
% Este es el archivo que se debe compilar.
% 
% Este template ha sido modificado y actualizado por Eduardo Castro y Roosevelt  en base a lo trabajado por Víctor Murray, Oscar Ramos y Juan Carlos Barbaran.
%
% Última actualización: Mayo, 2022

\documentclass[a4paper,12pt,oneside]{tesisutec}

\selectlanguage{spanish}

%% Paquetes
\usepackage[utf8]{inputenc}
\usepackage[square, numbers, comma, sort&compress]{natbib}
% Librería de idioma
\usepackage[spanish]{babel}
% Librería para posicionamiento
\usepackage{float}
% Librerías para insertar códigos
\usepackage[spanish,onelanguage,ruled,vlined]{algorithm2e}
\usepackage{verbatim} 
% Librería para híper vínculos
\usepackage{hyperref}
 % Librería necesaria para arreglar el orden de referencias .com
\usepackage{notoccite}

% Incluir acá los paquetes adicionales que deseas. 
% Ubicación de los imágenes.
\graphicspath{images/}

\makeatletter
%  missing \algbackskip
\def\algbackskip{\hskip-\ALG@thistlm}
\makeatother

\hypersetup{urlcolor=blue, colorlinks=true}

\begin{document}

\frontmatter

\department{}
\degree{bachiller }
\major{Ingeniería Mecatrónica}

\title{Monitoreo de granjas de jaula para salmonicultura usando visión computacional}

\author{Jim Elashir Fabián Gonzales \orcid{0000-0000-0000-0000}} % Es obligatorio agregar ORCID del alumno
\supervisor{Oscar E. Ramos \orcid{0000-0000-0000-0000}} % Es obligatorio agregar ORCID del asesor

\date{2023}

\maketitle

\setstretch{1.5}


% \input{encabezados/dedicatoria}
% \input{encabezados/agradecimientos}

\tableofcontents

\newpage

% \listoftables

\newpage

% \listoffigures

\addtocontents{toc}{\vspace{1.5em}}

%% ============================================================================
\mainmatter
\pagestyle{fancy}

\customchapter{RESUMEN}

La acuicultura es una industria emergente con mayor presencia en el mercado, siendo el salmón uno de los peces más populares. Existen diferentes métodos y estructuras de cultivo, siendo uno de los más utilizados los criaderos de peces en jaulas hechas de redes ubicadas en mar abierto. A medida que aumenta el volumen de producción, se requiere automatización en su mantenimiento. Uno de los principales desafíos es la detección de agujeros y suciedad, y el método común es la inspección de buzos marinos. Esta tesis presenta una investigación práctica para el monitoreo en tiempo real de los criaderos de peces utilizando herramientas de visión por computadora e inteligencia artificial. El objetivo principal es desarrollar un sistema capaz de identificar y cuantificar la integridad de la red y el nivel de suciedad en las jaulas. En primer lugar, para los agujeros en la red, se desarrolla un método de aprendizaje profundo utilizando un algoritmo modificado de Mask R-CNN para lograr una detección robusta y mejorada. En segundo lugar, para la presencia de suciedad, se utilizan diferentes filtros de procesamiento de imágenes para obtener un área estimada creada por esta. Además, para una visualización completa de la red de peces, se crea una reconstrucción utilizando monocular SLAM visual inercial de presión. Los resultados preliminares están relacionados con el algoritmo de aprendizaje profundo, donde se crea un conjunto de datos de redes de peces utilizando diferentes imágenes de una búsqueda en la web.


\noindent \textbf{Palabras clave:}\\
\noindent  Visual inertial pressure monocular SLAM; Robotics; Deep Learning; Computer vision; Aquaculture
\customchapter{ABSTRACT} 

\begin{center}
\large \vspace{-1.5cm} \textbf{Monitoring Cage Farms for Salmon Aquaculture Using Computer Vision}
\end{center}
Aquaculture is an emerging industry with more presence in the market, being the salmon one of the most popular fish. There are different farming methods and structures, in which one of the most used is fish cage farms made by nets located at open sea. Since the production volumes increases, automation in its maintenance is needed. One of the main challenges is the detection of holes and dirt, and the common method is the inspection of marine divers. This thesis presents a practical research for real-time monitoring of fish cage farms using computer vision and AI tools. The main objective is to develop a system capable of identifying and quantifying the integrity of the net and the level of dirt in the cages. Firstly, for the net holes, a Deep Learning method using Mask R-CNN modified algorithm is developed for a robustness and better detection. Secondly, for dirt presence, different image processing filters are used in order to obtain an estimated created area of it. Moreover, for a complete visualization of the fish net, a reconstruction is created using visual inertial pressure monocular SLAM. The preliminary results are related to the DL algorithm , where a dataset of fish nets are created using different imagen from a web search.

\noindent \textbf{Keywords:}\\
\noindent Visual inertial pressure monocular SLAM; Robotics; Deep Learning; Computer vision; Aquaculture


 
\customchapter{INTRODUCCIÓN} 

\introsection{Presentación del tema de investigación}

El crecimiento de las granjas de crianza de salmones en la industria acuícola requiere de soluciones más escalables y precisas para medir parámetros de sus redes. Es por ello que se utilizan robots submarinos para la inspección y monitoreo de ellas; sin embargo, falta desarrollar distintas herramientas que mejoren la adquisición de datos. El campo de la visión computacional abre un abanico de herramientas que, al integrarse, se logra conseguir mayor información de las redes. 

\introsection{ Descripción de la situación problemática}

Debido al exceso de pesca de muchas especies marinas en la industria pesquera, ha surgido una industria que sustituye la pesca por la crianza de ellas, llamada acuícola. A lo largo de los años ha estado creciendo a nivel global, y, tanto países como especies, resaltan unas más que otras. Según el informe del estado mundial de la pesca y acuicultura al 2022 realizada por la FAO \cite{FAO}, esta industria produjo 281 300 millones de USD el 2020, teniendo un crecimiento del 500 \% desde 1990. Latinoamérica forma parte con 14 000 millones de USD, mientras que Perú 462 millones. 

Sin embargo, la mayor participación lo lleva Chile con 4 800 millones y con crecimiento del 47.6 \% para el 2030. Aún más, junto a Noruega son los 2 países de mayor exportación de esta industria con la crianza del salmón en mar abierto, con el uso de granjas de jaula. Estas son redes suspendidas de gran diámetro situados en cuerpos de agua como ríos, lagos o mares. Las medidas rondan entre 60 y 120 metros de longitud, con un diámetro de 30 a 40 metros por cada granja. 

Ante la gran cantidad de granjas por empresa, se generan muchas problemáticas para conocer su estado como de los salmones\cite{FAO}. Algunas de ellas son: contaminación del medio ambiente, por exceso de alimentos y desechos orgánicos; tasa de mortalidad, por enfermedades o parásitos al dejar los peces muertos en el fondo; y mantenimiento y monitoreo de las granjas, por huecos en las redes y el porcentaje de suciedad. Este último es muy importante, ya que el reconocimiento tardío de huecos trae como consecuencia posible fuga del pescado. Asimismo, el porcentaje de suciedad en las redes da a conocer si hay un flujo de agua dentro de ella, asegurando calidad de agua y menor riesgo a contaminación al pescado. 

Estas problemáticas, para ser resueltas, necesitan de información para tomar decisiones. Tanto el área a rodear de las granjas y la cantidad de ellas por empresa resulta muy trabajoso y riesgoso para que personas lo puedan realizar. Es por ello que se han estado implementando el uso de ROVs, o vehículos operados remotamente, que realizan el mismo trabajo pero con más ventajas: reducción del riesgo laboral para tomar datos, datos más precisos y en tiempo real. En otras palabras, un robot logra tomar información cuantitativa que puede ser analizado de mejor manera, tanto en el momento de la medición como comparando con datos históricos.

Los instrumentos que utilizan estos robots son sensores que permiten recoger información del entorno. Uno de ellos es la cámara, y esta suele ser estéreo dado que es una forma de medir la profundidad de un objeto respecto al robot. A través de el se puede obtener información visual de las redes de las granjas, dando una herramienta de medición para la problemática del monitoreo de ellas. Este sensor permite reconocer a través de imágenes cómo se ven las granjas; sin embargo, este reconocimiento no debe ser manual necesariamente. Para un procesamiento automático de imágenes se tiene el campo de visión computacional que ofrece herramientas para la detección de objetos y parámetros que se deseen reconocer. 

\introsection{Formulación del problema}

Englobando lo mencionado en la sección anterior, el monitoreo de redes es una problemática que tienen las granjas y que destaca por sus consecuencias para la empresa acuícola. Ante la cantidad de información que debe ser recopilada por la magnitud de ellas se tienen los ROVs que logran obtener información visual. Sin embargo, hace falta automatizar el reconocimiento de huecos y la cantidad de suciedad que tienen las redes. Es por ello que surge la necesidad de una solución tecnológica que realice una medición de estos datos para la reducción de estas problemáticas.

Por otro lado, partiendo del valor de mercado de esta industria y la problemática encontrada, esta investigación será práctica. Es por ello que, con el fin de generar un mayor impacto, la investigación se centrará en estas granjas de jaula para salmones.  

\introsection{Objetivos de investigación}

El objetivo principal de este proyecto de investigación práctica es \textbf{desarrollar e implementar un sistema para monitoreo de granjas de jaula para salmonicultura usando visión computacional en un ROV.} Para poder alcanzarlo, se plantean los siguientes objetivos específicos:

\begin{itemize}
    \item Implementar algoritmos de visión computacional y Machine Learning / Deep Learning para reconocimiento de objetos usando un ROV. A través del estado del arte se conocerá los instrumentos necesarios así como las tecnologías actuales para fusionarlos.
    \item Verificar el algoritmo en el programa de simulación UV simulator. Este programa facilita el uso de un ROV genérico en el que se le puede insertar los sensores necesarios para el algoritmo planteado. Asimismo se pueden generar cuerpos rígidos, por lo que también se logrará simular una granja de jaula.
    \item Implementar el sistema de monitoreo moviendo el ROV en un ambiente controlado. Este consta de una piscina donde se colocará una pequeña jaula que representará una granja de jaula.
    \item Optimizar el algoritmo para mejores resultados. En función a los parámetros que se toman en el algoritmo, estos pueden variar con el fin de tener un mejor resultado.
\end{itemize}

\introsection{Justificación}

Las problemáticas mencionadas anteriormente fueron obtenidas de varias fuentes que lograron complementarse. Uno de ellas fue por medio de entrevistas de un grupo de investigación llamado Blume, creado dentro de la universidad UTEC. Estas fueron entregadas de manera anónima, es decir, solo resúmenes de las entrevistas \cite{Entrevistas_Blume}. Dentro de las respuestas de las empresas, se resalta la existencia parcial de las herramientas usadas en la actualidad. Utilizan ROVs para el monitoreo de redes, pero no tienen una solución que detecte deficiencias o parámetros en las redes. Por otro lado, vieron una gran oportunidad ya que, en la problemática de la tasa de mortalidad, no encuentran en el mercado un ROV capaz de levantar distintos cuerpos (como salmones) sin perder el control. 

\introsection{Alcance y limitaciones / restricciones} %[opcional]

Los alcances de este proyecto servirán para comparar el resultado obtenido con el deseado inicialmente. En primer lugar, desarrollar línea de base para comprender mejor los requisitos específicos de visión computacional para estas aplicaciones. En segundo lugar, implementar un sistema para monitoreo usando la menor cantidad de hardware posible. En tercer lugar, el sistema de monitoreo no implica el desarrollo de una base de datos ni HMI para almacenamiento y visualización de data, así como rutas autónomas. Finalmente, el sistema se probará en un ambiente controlado. Esto significa que los resultados no trascienden más allá de este. 

Seguidamente se presentan las limitaciones y restricciones que se tomarán en cuenta a lo largo del proyecto. Por un lado, para la implementación se debe obtener un ROV con los sensores suficientes. Luego, el acceso a aquellos ambientes con granjas de jaula a escala. Si bien sería un evento.
\chapter{REVISIÓN CRÍTICA DE LA LITERATURA}

Para desarrollar un sistema tecnológico se debe conocer qué existe actualmente en el mercado, tanto en hardware como en software. En este sentido, se buscará entre las distintas empresas qué servicios realizan para las granjas de jaula, qué robots utilizan y qué métodos utilizan relacionados la visión computacional. Seguidamente, con el fin de simular correctamente el sistema desarrollado, se debe identificar qué software de simulación es el mejor para crear ambientes parecidos. 

\section{Empresas que genera soluciones en acuicultura}
La tecnología para la acuicultura es desarrollado por empresas contratistas que ofrecen distintos servicios. Cada una tiene un método distinto de ofrecerlos, ya sea con distintos equipos integrados o robots submarinos que lo simplifican o complementan. De acuerdo al sondeo realizado (ver anexo A), los servicios para las redes son los siguientes:

\subsection{Inspección de redes y sistemas de amarre}
Empresas como Qysea, AKVA group \cite{AKVAgroup} o UCO utilizan robots submarinos ROV (Vehículo Operado Remotamente) capaces de navegar alrededor de las granjas. Los servicios implican un operador in-situ manejando el ROV. Dentro de esta inspección, algunas empresas como Deep Trekker \cite{DeepTrekker}, Blue-eye robotics han desarrollado algoritmos de reconocimiento de huecos. Aún más, con sus ROVs pueden parchar tales huecos de manera rápida y simple. Una desventaja de estos servicios es que no realizan operaciones ni análisis autónomas, es decir, no utilizan herramientas de navegación autónoma o visión computacional respectivamente. 

\begin{figure} [!h]
    \begin{center}
    \begin{tabular}{ccc}
    \includegraphics[height=3cm]{images/Qysea_1} &
    \includegraphics[height=2cm]{images/AKVA_1} &
    \includegraphics[height=3cm]{images/Deep_1} \\
    (a) & (b) & (c)
    \end{tabular}
    \caption{\label{fig:empresas_manual}(a) ROV de Qysea para reparación de huecos. (b) Robot de AKVA para limpieza de redes. (c) ROV de Deep Trekker para inspección manual de redes.}
    \end{center}
\end{figure}


En contraste con las empresas mencionadas, InnovaSea \cite{InnovaSea} ofrece un sistema completo conformado por distintos equipos entre las granjas que miden parámetros del agua, nivel de alimentación, comportamiento y salud del pescado. Posee un dispositivo entre las conexiones cada granja, controlando el sistema de amarre. Sin embargo, no ofrece una solución para la inspección de redes. Cabe resaltar que tampoco utiliza ROVs, mas sí dispositivos que, para otros parámetros, realiza mediciones de forma estática. Para análisis de estas mediciones, a comparación de las empresas anteriormente mencionadas, sí utilizan herramientas de visión computacional para obtener la biomasa del pescado, control de alimentación, entre otros. 

\begin{figure} [!h]
    \begin{center}
    \includegraphics[height=4cm]{images/InnovaSea_1} 
    \caption{\label{fig:InnovaSea_sistema}Distribución de equipos de Innova Sea alrededor de las granjas.}
    \end{center}
\end{figure}


\subsection{Limpieza de redes}
Empresas como Autobots, Watbots o Remora robotics desarrollan robots submarinos con instrumentos que quitan todo material en la red. Los 3 ofrecen una limpieza autónoma,  Watbots y Remora pueden detectar obstáculos y esquivarlos, y Watbots también detecta huecos \cite{Watbots}. Cabe resaltar que estos robots solo poseen un diseño para limpieza de redes, sin la capacidad de ser modular. Asimismo, las empresas se enfocan en realizar servicios en las redes, mas no en las demás partes de la granja. Por este motivo no se le puede agregar una garra para arreglar el hueco, o una cámara para observar el comportamiento del pescado. 


% \cite{Reumann2012}.
%
% \begin{equation}
%   \label{eq:1}
%   a+b=\sqrt{\frac{4}{3}},
% \end{equation}
% donde $a$ y $b$ son escalares.

\section{Métodos de visión computacional}
Los métodos encontrados se basaron de documentos publicados de investigación, ya que las empresas no especifican el software que utilizan. Dado que el objetivo de esta investigación está relacionada con visión computacional, se han encontrado distintos métodos para cada parámetro por detectar en las redes. Estos se dividen en 2: detección de huecos y estimación de suciedad.

\subsection{Detección de huecos}



\subsection{Estimación de suciedad}

\section{Simuladores de sistemas mecatrónicos}

% Entre las causas se pueden considerar la falta de tiempo asignado al taller,
% las limitaciones en cuanto a laboratorios, el poco personal, la falta de una
% correcta selección de los contenidos, así como, una calificación que asegure el
% cumplimiento de los objetivos. Un ejemplo de cómo poner una figura se muestra en la Fig. \ref{fig:diagram2}, y es importante recordar la relación mostrada en \eqref{eq:1}.


% \begin{figure} 
% \begin{center}
% \begin{tabular}{cc}
% \includegraphics[height=3cm]{images/logo_utec.png} &
% \includegraphics[height=2cm]{images/logo_utec.png} \\
% (a) & (b)
% \end{tabular}
% \caption{\label{fig:diagram2} . (a) Logo en tamaño de 3 centímetros. (b) Logo en tamaño de 2 centímetros.}
% \end{center}
% \end{figure}

% \begin{table}[H]
%     \centering
%     \begin{tabular}{c|c}
%         Tiempo($s$) & Distancia($m$) \\
%         \hline
%         10 & 23 \\
%         20 & 33 \\
%         30 & 43 \\
%         40 & 53 \\
%     \end{tabular}
%     \caption{Tiempos versus distancia.}
%     \label{tab:tiempo_versus_distancia}
% \end{table}
\chapter{MARCO TEÓRICO}

El marco teórico de este proyecto de investigación se centrará en los conceptos y técnicas fundamentales que subyacen en el algoritmo que se desarrollará. Este algoritmo integrará técnicas de aprendizaje profundo (Deep Learning), visión por computadora y SLAM (Simultaneous Localization and Mapping) para el monitoreo de granjas de jaulas de salmón.

\section{Aprendizaje Profundo (Deep Learning)}

El aprendizaje profundo es una sub categoría de la inteligencia artificial que se centra en algoritmos inspirados en la estructura y función del cerebro llamados redes neuronales artificiales. Estos algoritmos son capaces de aprender de grandes cantidades de datos y han demostrado ser extremadamente eficaces en muchas aplicaciones, incluyendo la visión por computadora, el procesamiento del lenguaje natural, y el reconocimiento del habla.

En este proyecto, se utilizará una arquitectura de red neuronal convolucional específica llamada Mask R-CNN con ResNET-101, RFP y DCN para la detección de huecos en las redes de las jaulas de salmón. Las redes neuronales convolucionales (CNNs) son una clase especial de redes neuronales que han demostrado ser muy eficaces en tareas de visión por computadora. 

\subsection{Mask R-CNN}

Mask R-CNN es un modelo utilizado para la detección y segmentación de objetos en imágenes. Está basado en Feature Pyramid Network (FPN) y una estructura de red neuronal ResNet101 \cite{matterport_maskrcnn_2017}. Esta técnica permite la segmentación de instancias, es decir, la identificación y clasificación de objetos individuales en una imagen. La segmentación de instancias es una tarea importante en la visión por computadora, ya que permite la identificación y clasificación de objetos individuales en una imagen. 


Se puede dividir en 2 etapas: generación de la región deseada (ROI), y clasificación y segmentación. La primera etapa es una arquitectura de tipo backbone CNN, donde se extrae las características (o features) de la imagen. La segunda etapa, obtiene los bounding boxes de la detección para luego extraer únicamente su forma \cite{cite:Zhang}. El output de este algoritmo se puede apreciar en la imagen \ref{fig:mask_crnn}.

\begin{figure} [!h]
    \begin{center}
    \includegraphics[height=6cm]{images/mask_crnn.png} 
    \caption{\label{fig:mask_crnn} Ejemplo de lo obtenido del método MASK CRNN en \cite{matterport_maskrcnn_2017}.}
    \end{center}
\end{figure}


\subsection{Redes residuales ResNET-101}

Las redes residuales (ResNet) son redes neuronales que se basan en el concepto de aprendizaje residual. Estas redes están compuestas por bloques de construcción que incluyen capas convolucionales y conexiones de atajo. La idea principal detrás de las ResNet es permitir que las redes sean más profundas sin sufrir degradación en el rendimiento. Esto se logra mediante la introducción de conexiones de atajo que saltan una o más capas, lo que permite que la información fluya directamente a través de la red. 

ResNet-101 es una variante específica de las redes residuales que consta de 101 capas. Esta arquitectura se destaca por utilizar bloques residuales como sus componentes principales. Cada bloque residual está compuesto por varias capas convolucionales y de activación que se organizan en una estructura en cascada. La salida de un bloque residual se convierte en la entrada del siguiente bloque. Sin embargo, lo que diferencia a las ResNet de otras arquitecturas es la introducción de conexiones de atajo o conexiones residuales \cite{cite:Kaiming}.

\subsection{Feature Pyramid Networks (FPN)}

Feature Pyramid Networks (FPN) es una arquitectura que ha demostrado una mejora significativa como un extractor de características genérico en varias aplicaciones, incluyendo la detección de objetos, segmentación semántica y segmentación de instancias. FPN está diseñada para abordar el desafío de detectar objetos en diferentes escalas en una imagen. El enfoque tradicional para la detección de objetos involucra el uso de un mapa de características de una sola escala, lo cual no es efectivo para detectar objetos de diferentes tamaños. FPN, por otro lado, construye una pirámide de características que posee una rica semántica en todos los niveles y se construye rápidamente a partir de una sola escala de imagen de entrada \cite{cite:Tsung}.

La arquitectura de FPN combina características de baja resolución y alta semántica con características de alta resolución y baja semántica a través de una vía descendente y conexiones laterales. Esto resulta en una pirámide de características que tiene una fuerte semántica en todas las escalas y puede ser utilizada para reemplazar pirámides de imágenes con características sin sacrificar el poder representativo, la velocidad o la memoria. El éxito de FPN en la detección de objetos resalta la importancia de utilizar redes neuronales convolucionales profundas para construir pirámides de características que puedan detectar eficazmente objetos en diferentes escalas.


\section{Visión por Computadora}

La visión por computadora es una disciplina que se centra en enseñar a las máquinas a 'ver' e interpretar imágenes y videos de la misma manera que los humanos. En este proyecto, se utilizarán técnicas de visión por computadora para identificar áreas de la red que están sucias y estimar el porcentaje total de suciedad. Parte de las herramientas usadas son distintos procedimientos de imágenes digitales, los cuales serán explicados con información extraída de \cite{cite:Gonzalez}.

\subsection{Método de Otsu}

El método de Otsu es una técnica de umbralización que se utiliza para convertir una imagen en escala de grises en una imagen binaria. Este método selecciona el umbral que minimiza la varianza intra clase de los píxeles de la imagen. En este proyecto, el método de Otsu se utilizará para segmentar la imagen de la red y facilitar la identificación de las áreas sucias.

\subsection{Histograma de Gradientes Orientados (HOG)}

El Histograma de Gradientes Orientados (HOG) es una técnica que se utiliza para la extracción de características en imágenes. Esta técnica cuenta las ocurrencias de los gradientes de intensidad de la imagen en direcciones orientadas. Los HOG son muy eficaces para la detección de formas y estructuras en las imágenes. En este proyecto, se utilizarán los HOG para identificar las áreas de la red que están sucias.


\section{SLAM Visual-Inercial-Presión}

SLAM (Simultaneous Localization and Mapping) es una técnica de Computer vision que permite a un robot mapear su entorno mientras se localiza en él. En este proyecto, se utilizará una versión de SLAM que combina datos de una cámara monocular, un sensor inercial (IMU) y un sensor de presión para la reconstrucción 3D de las granjas de jaulas de salmón. Para la explicación de este campo se extraerá información de \cite{cite:Gao} y \cite{cite:Ferrera}.

El SLAM visual-inercial es una variante de SLAM que utiliza datos de una cámara y un sensor inercial para estimar la trayectoria del robot y el mapa del entorno. Esta técnica es muy eficaz en entornos donde el GPS no está disponible o es poco fiable, como es el caso de las granjas de jaulas de salmón submarinas. La adición de un sensor de presión permite una estimación más precisa de la profundidad.
% \input{secciones/capitulo3}
\chapter{MARCO METODOLÓGICO} \label{MARCO_METODOLOGICO}

Esta investigación práctica se basa en la implementación de un algoritmo que utiliza técnicas de Deep Learning, SLAM visual-inercial-presión y procesamiento de imágenes para monitorear las redes de las granjas de jaulas de salmón. A continuación, se detalla la metodología que se seguirá para cada uno de estos componentes. 

\section{Desarrollo del algoritmo de Deep Learning}

El primer paso en el desarrollo del algoritmo de Deep Learning será la recopilación de un conjunto de datos de imágenes de redes de granjas de jaulas de salmón para crear el dataset. Estas imágenes serán etiquetadas manualmente para indicar la ubicación de los huecos en la red. Este conjunto de datos se dividirá en un conjunto de entrenamiento y un conjunto de prueba.

Para el entrenamiento del modelo de Deep Learning, se utilizará una arquitectura de red neuronal convolucional (CNN) basada en Mask CRNN presentada en \cite{cite:Zhang}. Esta arquitectura ha demostrado ser eficaz en tareas de reconocimiento de imágenes y se espera que sea capaz de identificar los huecos en las redes de las granjas de jaulas. Durante el entrenamiento, se utilizarán técnicas de aumento de datos para mejorar la robustez del modelo y evitar el sobre ajuste. 

Una vez que el modelo haya sido entrenado y validado, se integrará en un robot submarino. Para ello, se utilizará un sistema operativo robótico (ROS 2) que permitirá la comunicación entre el modelo de Deep Learning y el hardware del robot. Este sistema permitirá que el modelo de Deep Learning procese las imágenes capturadas por una cámara en tiempo real y detecte los huecos en la red.

\section{Aplicación de técnicas de procesamiento de imágenes}

Para la identificación de áreas sucias en la red, se utilizarán técnicas de procesamiento de imágenes presentadas en \cite{cite:Zhao}. En primer lugar, se utilizará el método de Otsu para convertir las imágenes en escala de grises a imágenes binarias. Este método selecciona un umbral que minimiza la varianza intra clase en la imagen, lo que resulta en una segmentación efectiva de las áreas sucias y limpias de la red.

Una vez que se ha obtenido la imagen binaria, se utilizará el Histograma de Gradientes Orientados (HOG) para extraer características de la imagen. El HOG es una técnica que cuenta las ocurrencias de las orientaciones de los gradientes en porciones de la imagen. Estas características serán útiles para identificar las áreas sucias de la red, ya que las áreas sucias y limpias tendrán diferentes distribuciones de orientaciones de gradientes.

Finalmente, se calculará el porcentaje de suciedad en la red contando el número de píxeles en las áreas identificadas como sucias y dividiéndolo por el número total de píxeles en la imagen. Este porcentaje de suciedad será una medida cuantitativa de la limpieza de la red.

\section{Aplicación de SLAM visual-inercial-presión} \label{VIP_SLAM}

La realizará una simulación e implementación del SLAM visual-inercial-presión presentada en \cite{cite:Ferrera}, el cual se realizará en tres etapas. En la primera etapa, se implementará un SLAM visual utilizando una cámara monocular. Se utilizará el algoritmo de SLAM visual ORB-SLAM2 \cite{murTRO2015} y se adaptará para trabajar con las imágenes de una cámara Kinect junto a un turtlebot Waffle pi. La simulación se realizará en ROS Noetic con Gazebo y con RViz la reconstrucción. Este algoritmo permitirá construir un mapa 3D de su entorno y localizarse dentro de este mapa. 

Cabe resaltar que en \cite{cite:Ferrera} se implementó UW-VO SLAM. Sin embargo, no se ha encontrado un repositorio de código abierto. Por tal motivo, dentro de la comparación que se realiza junto a otros métodos de monocular SLAM, ORB-SLAM es el segundo con menor error. 

En la segunda etapa, se añadirá la información de presión al algoritmo de SLAM visual. Para ello, se utilizará un sensor de presión que se incorporará en un sistema mecatrónico junto a la cámara. La información de presión se utilizará para mejorar la estimación de la profundidad en el algoritmo de SLAM. Esto permitirá conocer su profundidad en el agua, lo que es crucial para la navegación submarina.

En la tercera etapa, se añadirá la información inercial al algoritmo de SLAM. Para ello, se utilizará un IMU que se instalará al sistema mecatrónico. La información inercial se utilizará para mejorar la estimación de la orientación y la velocidad en el algoritmo de SLAM. Esto permitirá conocer su orientación y velocidad del sistema.

\chapter{RESULTADOS PRELIMINARES}

Los resultados preliminares se han desarrollado con el fin de cumplir con el primer objetivo secundario, diseñar e implementar algoritmos de visión computacional y aprendizaje automático capaces de reconocer y cuantificar agujeros de red y nivel de suciedad. Como fue presentado en el capítulo \ref{MARCO_METODOLOGICO} en el inciso \ref{VIP_SLAM}, se simulará SLAM visual-inercial-presión con 3 etapas. Para la primera, uso de SLAM visual monocular con ORB-SLAM2.

Dentro del simulador Gazebo se utilizó un Turtlebot3, la cámara de un Kinect y un sensor IMU. A este último sensor se agregó un filtro de Kalman extendido para reducción de error de medición. Asimismo se realizó una malla de 1.75 m por 1 m de diámetro como se muestra en la figura \ref{fig:malla}.

\begin{figure} [!h]
    \begin{center}
    \includegraphics[height=6cm]{images/malla} 
    \caption{\label{fig:malla}Elaboración propia de una malla.}
    \end{center}
\end{figure}

Una vez el algoritmo esté corriendo, se obtuvo el diagrama de tópicos y nodos mostrados en la figura \ref{fig:rqt_graph}. Se puede apreciar que el algoritmo como nodo 'orb\_slam2\_mono' usa información del sensor IMU con filtro Kalman extendido 'nodo\_imu\_ekf' publicado en el tópico 'tf'. Del mismo modo, utiliza las imágenes publicadas en el tópico 'image\_raw' del Kinect. 

\begin{figure}[!h]
    \begin{center}
    \includegraphics[height=4cm]{images/rqt_graph} 
    \caption{\label{fig:rqt_graph}Diagrama de tópicos y nodos en ROS.}
    \end{center}
\end{figure}

En la figura \ref{fig:sim_ros} se aprecia el entorno en Gazebo en (a), donde la luz tiene un tono azulado. En (b) en ell visualizador RViz se tiene el turtlebot moviendo, donde ya encuentra los keypoints de la red. 

\begin{figure} [!h]
    \begin{center}
    \begin{tabular}{cc}
    \includegraphics[height=5cm]{images/gazebo_1}
    \includegraphics[height=5cm]{images/rviz_1} \\
    (a) & (b)
    \end{tabular}
    \caption{\label{fig:sim_ros}Turtlebot en (a) Gazebo y (b) RViz en la búsqueda de keypoints.}
    \end{center}
\end{figure}

La reconstrucción en RViz se puede apreciar en la figura \ref{fig:sim_ros_1}. En (a) se puede apreciar la reconstrucción a partir de las imágenes captadas, mientras que en (b), los keypoints encontrados. En el primero se resalta que el robot no logra encontrar los keypoints de la parte posterior, y se asume que es por la confusión de los keypoints encontrados en la parte trasera de la malla respecto a dónde se encuentra. Asimismo, al intentar construir la parte trasera del robot se puede apreciar unas líneas faltantes, siendo estas la red de la parte posterior de la malla.

\begin{figure} [!h]
    \begin{center}
    \begin{tabular}{cc}
    \includegraphics[height=5cm]{images/rviz_2}
    \includegraphics[height=5cm]{images/rviz_3} \\
    (a) & (b)
    \end{tabular}
    \caption{\label{fig:sim_ros_1}Reconstrucción en RViz en (a) a partir de las imágenes, y en (b) a partir de los keypoints.}
    \end{center}
\end{figure}

Estos resultados se pueden alejar de la realidad, ya que la parte trasera de la red no se aprecia bajo el mar debido a la turbidez del agua, por lo que se propone para siguientes pruebas este cambio. Por otro lado, el Turtlebot, al no tener un grado de libertad en el eje z, no se puede elevar. Por ese lado, se buscará un robot como un dron u otro simulador capaz de utilizar otro robot robot con este grado de libertad.

% \input{secciones/conclusiones}
% \input{secciones/recomendaciones} % sección opcional

%% ============================================================================
\renewcommand{\bibname}{\hfill\Large\bf{REFERENCIAS BIBLIOGRÁFICAS}\hfill}

\bibliographystyle{IEEEtran} % Estableciendo el estilo de citas IEEE 

\bibliography{referencias} % Recibe las referencias de IEEE

\chapter*{\center \Large ANEXOS} 
\addcontentsline{toc}{section}{\bfseries ANEXOS} 
\markboth{ANEXOS}{ANEXOS} 

\newpage
\begin{center}
    {ANEXO A: Empresas que ofrecen servicios a empresas acuícolas} \label{empresas}
\end{center}

\begin{figure}[h!]
    \begin{center}
    \includegraphics[height=19cm]{images/Empresas.png} 
    \end{center}
\end{figure}

\end{document}
