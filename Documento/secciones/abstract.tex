\customchapter{ABSTRACT} 

\begin{center}
\large \vspace{-1.5cm} \textbf{Monitoring Cage Farms for Salmon Aquaculture Using Computer Vision}
\end{center}
Aquaculture is an emerging industry with more presence in the market, being the salmon one of the most popular fish. There are different farming methods and structures, in which one of the most used is fish cage farms made by nets located at open sea. Since the production volumes increases, automation in its maintenance is needed. One of the main challenges is the detection of holes and dirt, and the common method is the inspection of marine divers. This thesis presents a practical research for real-time monitoring of fish cage farms using computer vision and AI tools. The main objective is to develop a system capable of identifying and quantifying the integrity of the net and the level of dirt in the cages. Firstly, for the net holes, a Deep Learning method using Mask R-CNN modified algorithm is developed for a robustness and better detection. Secondly, for dirt presence, different image processing filters are used in order to obtain an estimated created area of it. Moreover, for a complete visualization of the fish net, a reconstruction is created using visual inertial pressure monocular SLAM. The preliminary results are related to the DL algorithm , where a dataset of fish nets are created using different imagen from a web search.

\noindent \textbf{Keywords:}\\
\noindent Visual inertial pressure monocular SLAM; Robotics; Deep Learning; Computer vision; Aquaculture


