\chapter{REVISIÓN CRÍTICA DE LA LITERATURA}

.

\section{Sección 1}

Lorem ipsum dolor sit amet, consectetur adipiscing elit. Cras scelerisque quis augue malesuada porttitor. Phasellus tincidunt ipsum quis metus pellentesque commodo. Phasellus vitae metus arcu. Pellentesque scelerisque ac nulla laoreet mollis. Vivamus blandit, velit eget convallis eleifend, nibh risus congue sapien, tristique bibendum justo lorem eget arcu. Quisque mollis tristique tellus, sit amet hendrerit purus lacinia id. Fusce sed nunc arcu. Phasellus porttitor convallis ipsum, sit amet tempus nulla mattis a \cite{Reumann2012}.

.

\section{Segundo subtítulo}

\subsection{División del segundo subtítulo}

La industria acuícola 
%
\begin{equation}
  \label{eq:1}
  a+b=\sqrt{\frac{4}{3}},
\end{equation}
%
donde $a$ y $b$ son escalares.

Entre las causas se pueden considerar la falta de tiempo asignado al taller,
las limitaciones en cuanto a laboratorios, el poco personal, la falta de una
correcta selección de los contenidos, así como, una calificación que asegure el
cumplimiento de los objetivos. Un ejemplo de cómo poner una figura se muestra en la Fig. \ref{fig:diagram2}, y es importante recordar la relación mostrada en \eqref{eq:1}.


\begin{figure} 
\begin{center}
\begin{tabular}{cc}
\includegraphics[height=3cm]{images/logo_utec.png} &
\includegraphics[height=2cm]{images/logo_utec.png} \\
(a) & (b)
\end{tabular}
\caption{\label{fig:diagram2}Scheme showing the architecture of a generic kinematic task. (a) Logo en tamaño de 3 centímetros. (b) Logo en tamaño de 2 centímetros.}
\end{center}
\end{figure}


\section{Tercer subtítulo}

\subsection{División del tercer subtítulo}

En la actualidad, las unidades de información hacen frente a muchos cambios
debido a los avances tecnológicos, explosión informativa, nuevos recursos y
soportes, por lo cual se implementan servicios innovadores que les permitan a
sus usuarios tener acceso a muchas fuentes de información. Para asegurar el
acceso y uso de los servicios los usuarios requieren poseer una serie de
habilidades que les permitan identificar, recuperar, manejar, discernir,
organizar, utilizar y comunicar la información de manera eficaz para la toma de
decisiones.

Entre las causas se pueden considerar la falta de tiempo asignado al taller,
las limitaciones en cuanto a laboratorios, el poco personal, la falta de una
correcta selección de los contenidos, así como, una calificación que asegure el
cumplimiento de los objetivos.

\begin{table}[H]
    \centering
    \begin{tabular}{c|c}
        Tiempo($s$) & Distancia($m$) \\
        \hline
        10 & 23 \\
        20 & 33 \\
        30 & 43 \\
        40 & 53 \\
    \end{tabular}
    \caption{Tiempos versus distancia.}
    \label{tab:tiempo_versus_distancia}
\end{table}