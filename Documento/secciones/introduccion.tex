\customchapter{INTRODUCCIÓN} 

\introsection{Presentación del tema de investigación}

El crecimiento de las granjas de crianza de salmones en la industria acuícola requiere de soluciones más escalables y precisas para medir parámetros de sus redes. Es por ello que se utilizan robots submarinos para la inspección y monitoreo de ellas; sin embargo, falta desarrollar distintas herramientas que mejoren la adquisición de datos. El campo de la visión computacional abre un abanico de herramientas que, al integrarse, se logra conseguir mayor información de las redes. 

\introsection{ Descripción de la situación problemática}

Debido al exceso de pesca de muchas especies marinas en la industria pesquera, ha surgido una industria que sustituye la pesca por la crianza de ellas, llamada acuícola. A lo largo de los años ha estado creciendo a nivel global, y, tanto países como especies, resaltan unas más que otras. Según el informe del estado mundial de la pesca y acuicultura al 2022 realizada por la FAO \cite{FAO}, esta industria produjo 281 300 millones de USD el 2020, teniendo un crecimiento del 500 \% desde 1990. Latinoamérica forma parte con 14 000 millones de USD, mientras que Perú 462 millones. 

Sin embargo, la mayor participación lo lleva Chile con 4 800 millones y con crecimiento del 47.6 \% para el 2030. Aún más, junto a Noruega son los 2 países de mayor exportación de esta industria con la crianza del salmón en mar abierto, con el uso de granjas de jaula. Estas son redes suspendidas de gran diámetro situados en cuerpos de agua como ríos, lagos o mares. Las medidas rondan entre 60 y 120 metros de longitud, con un diámetro de 30 a 40 metros por cada granja. 

Ante la gran cantidad de granjas por empresa, se generan muchas problemáticas para conocer su estado como de los salmones\cite{FAO}. Algunas de ellas son: contaminación del medio ambiente, por exceso de alimentos y desechos orgánicos; tasa de mortalidad, por enfermedades o parásitos al dejar los peces muertos en el fondo; y mantenimiento y monitoreo de las granjas, por huecos en las redes y el porcentaje de suciedad. Este último es muy importante, ya que el reconocimiento tardío de huecos trae como consecuencia posible fuga del pescado. Asimismo, el porcentaje de suciedad en las redes da a conocer si hay un flujo de agua dentro de ella, asegurando calidad de agua y menor riesgo a contaminación al pescado. 

Estas problemáticas, para ser resueltas, necesitan de información para tomar decisiones. Tanto el área a rodear de las granjas y la cantidad de ellas por empresa resulta muy trabajoso y riesgoso para que personas lo puedan realizar. Es por ello que se han estado implementando el uso de ROVs, o vehículos operados remotamente, que realizan el mismo trabajo pero con más ventajas: reducción del riesgo laboral para tomar datos, datos más precisos y en tiempo real. En otras palabras, un robot logra tomar información cuantitativa que puede ser analizado de mejor manera, tanto en el momento de la medición como comparando con datos históricos.

Los instrumentos que utilizan estos robots son sensores que permiten recoger información del entorno. Uno de ellos es la cámara, y esta suele ser estéreo dado que es una forma de medir la profundidad de un objeto respecto al robot. A través de el se puede obtener información visual de las redes de las granjas, dando una herramienta de medición para la problemática del monitoreo de ellas. Este sensor permite reconocer a través de imágenes cómo se ven las granjas; sin embargo, este reconocimiento no debe ser manual necesariamente. Para un procesamiento automático de imágenes se tiene el campo de visión computacional que ofrece herramientas para la detección de objetos y parámetros que se deseen reconocer. 

\introsection{Formulación del problema}

Englobando lo mencionado en la sección anterior, el monitoreo de redes es una problemática que tienen las granjas y que destaca por sus consecuencias para la empresa acuícola. Ante la cantidad de información que debe ser recopilada por la magnitud de ellas se tienen los ROVs que logran obtener información visual. Sin embargo, hace falta automatizar el reconocimiento de huecos y la cantidad de suciedad que tienen las redes. Es por ello que surge la necesidad de una solución tecnológica que realice una medición de estos datos para la reducción de estas problemáticas.

Por otro lado, partiendo del valor de mercado de esta industria y la problemática encontrada, esta investigación será práctica. Es por ello que, con el fin de generar un mayor impacto, la investigación se centrará en estas granjas de jaula para salmones.  

\introsection{Objetivos de investigación}

El objetivo principal de este proyecto de investigación práctica es \textbf{desarrollar e implementar un sistema para monitoreo de granjas de jaula para salmonicultura usando visión computacional en un ROV.} Para poder alcanzarlo, se plantean los siguientes objetivos específicos:

\begin{itemize}
    \item Implementar algoritmos de visión computacional y Machine Learning / Deep Learning para reconocimiento de objetos usando un ROV. A través del estado del arte se conocerá los instrumentos necesarios así como las tecnologías actuales para fusionarlos.
    \item Verificar el algoritmo en el programa de simulación UV simulator. Este programa facilita el uso de un ROV genérico en el que se le puede insertar los sensores necesarios para el algoritmo planteado. Asimismo se pueden generar cuerpos rígidos, por lo que también se logrará simular una granja de jaula.
    \item Implementar el sistema de monitoreo moviendo el ROV en un ambiente controlado. Este consta de una piscina donde se colocará una pequeña jaula que representará una granja de jaula.
    \item Optimizar el algoritmo para mejores resultados. En función a los parámetros que se toman en el algoritmo, estos pueden variar con el fin de tener un mejor resultado.
\end{itemize}

\introsection{Justificación}

Las problemáticas mencionadas anteriormente fueron obtenidas de varias fuentes que lograron complementarse. Uno de ellas fue por medio de entrevistas de un grupo de investigación llamado Blume, creado dentro de la universidad UTEC. Estas fueron entregadas de manera anónima, es decir, solo resúmenes de las entrevistas \cite{Entrevistas_Blume}. Dentro de las respuestas de las empresas, se resalta la existencia parcial de las herramientas usadas en la actualidad. Utilizan ROVs para el monitoreo de redes, pero no tienen una solución que detecte deficiencias o parámetros en las redes. Por otro lado, vieron una gran oportunidad ya que, en la problemática de la tasa de mortalidad, no encuentran en el mercado un ROV capaz de levantar distintos cuerpos (como salmones) sin perder el control. 

\introsection{Alcance y limitaciones / restricciones} %[opcional]

Los alcances de este proyecto servirán para comparar el resultado obtenido con el deseado inicialmente. En primer lugar, desarrollar línea de base para comprender mejor los requisitos específicos de visión computacional para estas aplicaciones. En segundo lugar, implementar un sistema para monitoreo usando la menor cantidad de hardware posible. En tercer lugar, el sistema de monitoreo no implica el desarrollo de una base de datos ni HMI para almacenamiento y visualización de data, así como rutas autónomas. Finalmente, el sistema se probará en un ambiente controlado. Esto significa que los resultados no trascienden más allá de este. 

Seguidamente se presentan las limitaciones y restricciones que se tomarán en cuenta a lo largo del proyecto. Por un lado, para la implementación se debe obtener un ROV con los sensores suficientes. Luego, el acceso a aquellos ambientes con granjas de jaula a escala. Si bien sería un evento.