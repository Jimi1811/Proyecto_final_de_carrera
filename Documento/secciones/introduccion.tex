\customchapter{INTRODUCCIÓN} 

\introsection{Presentación del tema de investigación}

La industria acuícola enfocada en los salmones, o salmonicultura, utilizan granjas de jaula de gran volumen ubicados en mar abierto. Identificar constantemente el estado de estas redes genera un impacto en la salud y pérdida de esta especie. Es por ello que se utilizan distintos sistemas mecatrónicos para la inspección y monitoreo de estas granjas; sin embargo, falta desarrollar herramientas que mejoren la adquisición y procesamiento de estos datos. El campo de la visión computacional abre un abanico de estas que, al integrarse, consiguen interpretar mayor información de las redes. 
 
% En primer lugar, el flujo del agua dentro de las granjas afecta en el crecimiento y la salud del pescado. En segundo lugar, reconocer y sellar los huecos en las redes evita una fuga del salmón.

\introsection{ Descripción de la situación problemática}

Debido al exceso de pesca de muchas especies marinas en la industria pesquera ha surgido una que la sustituye por la crianza de estas especies, la industria acuícola. A lo largo de los años esta industria ha estado creciendo a nivel global. Según el informe del estado mundial de la pesca y acuicultura al 2022 realizada por la FAO \cite{FAO}, esta industria produjo 281 300 millones de USD el 2020, teniendo un crecimiento del 500 \% desde 1990. Dentro de las especies con mayor volumen del mercado se encuentran la carpa herbívora y el salmón. China y Vietnam son los países con mayor producción de la carpa, mientras que Chile y Noruega del salmón. 

Uno de los métodos de crianza que más se utiliza para una mayor escalabilidad son las granjas de jaula. Estas son redes con cuerpo cilíndrico de gran diámetro suspendidas en cuerpos de agua como lagos o mares. Las empresas acuícolas contratan terceros que diseñan estas redes, siendo de distintos materiales y dimensiones. De acuerdo a un sondeo realizado en estas empresas (ver anexo A), las medidas rondan aproximadamente entre 60 y 200 metros de longitud, con un diámetro de 30 a 60 metros por cada granja. Ante la gran cantidad de granjas por empresa se generan muchas problemáticas para obtener información actualizada y rápida de estas.

Conocer el estado de las granjas tendrá un impacto en 3 áreas: salud del salmón, la salida del salmón o entrada de un depredador, y estabilidad estructural. En primer lugar, es de vital importancia mantener un flujo del agua dentro de la infraestructura, ya que permite la oxigenación del ambiente del pescado. Este factor influye en muchos parámetros del agua que afectan directamente al crecimiento del pescado. De la misma forma, se puede evitar la acumulación de distintas bacterias o micro algas, siendo estas perjudiciales en la salud y mortalidad del salmón. 

En segundo lugar, un hueco en la red puede ser causado por un incidente en el uso de equipos o la aparición de depredadores. Identificarlos en el menor tiempo posible evita una posible pérdida del pescado, tanto por la entrada de un depredador como por la fuga de los mismos. En tercer lugar, estas redes suelen ser colocadas por grupos, estando conectadas una con la otra y ancladas para que no muevan. Este sistema de amarre debe ser monitoreado, con el fin de evitar fallas estructurales o pérdidas de redes enteras. 

Estas problemáticas, para ser controladas, necesitan de información para tomar decisiones. Es por ello que se han estado desarrollando distintos sistemas mecatrónicos, tanto para la medición del entorno como para la actuación del control. Un robot submarino o un sistema estático de cámaras y sensores logran tomar información cuantitativa que puede ser analizada de mejor manera. Este tipo de soluciones son presentadas como servicios, donde cada empresa tiene una aplicación distinta. Sin embargo, como se verá más adelante en el capítulo 1, no muchas empresas consideran una solución automatizada para identificar los puntos mencionados anteriormente. 

\introsection{Formulación del problema}

Englobando lo mencionado en la sección anterior, el reconocimiento de huecos en las redes, el porcentaje de suciedad en ellas y el estado de su sistema de amarre son de vital importancia. Si bien existen servicios que actúan para erradicar el problema, no hay herramientas que se enfoquen en la automatización de la medición constante de estos parámetros de forma escalable. Es por ello que surge la necesidad de un sistema tecnológico que realice una recopilación e interpretación de estos datos para una rápida adquisición. Por otro lado, partiendo del valor de mercado de esta industria y la problemática encontrada, esta investigación será práctica. Es por ello que, con el fin de generar un mayor impacto, la investigación se centrará en estas granjas de jaula para salmones.  

\introsection{Objetivos de investigación}

El objetivo principal de este proyecto de investigación práctica es \textbf{desarrollar e implementar un sistema basado en visión por computadora para el monitoreo en tiempo real de granjas de jaulas de salmón, capaz de identificar y cuantificar la integridad de la red y el nivel de suciedad.} Para poder alcanzarlo, se plantean los siguientes objetivos específicos:

\begin{itemize}
    \item Diseñar e implementar algoritmos de visión computacional y aprendizaje automático capaces de reconocer y cuantificar agujeros de red y nivel de suciedad.
    \item Validar e integrar los algoritmos validados en un sistema de monitoreo y probar su rendimiento en un entorno controlado y reducido que simula una granja de jaula de salmón.
    \item Optimizar el sistema en función de los resultados de las pruebas in situ para mejorar el rendimiento y la confiabilidad.
\end{itemize}

\introsection{Justificación}

Las problemáticas mencionadas anteriormente fueron obtenidas de varias fuentes que lograron complementarse. Uno de ellas fue por medio de entrevistas de un grupo de investigación llamado Blume, creado dentro de la universidad UTEC. Estas fueron entregadas resumiendo los problemas que fueron reconocidos de empresas acuícolas en Chile \cite{Entrevistas_Blume}. Por otro lado, se realizó una búsqueda extensiva de los servicios que ofrecen empresas contratistas (ver anexo A). Esto permitió tener un panorama actualizado de la tecnología que se usa en la actualidad, donde se aprecian sistemas integrados mecatrónicos y el uso de inteligencia artificial.

\introsection{Alcance y limitaciones / restricciones} %[opcional]

Para los alcances iniciales se tiene, en primer lugar, desarrollar una línea de base para comprender mejor los requisitos específicos de visión computacional para estas aplicaciones. Esta base se desarrollará al realizar el capítulo 1, donde se conocerá qué métodos se usan y cuáles son los de mejor resultado para las aplicaciones planteadas. En segundo lugar, implementar un sistema para monitoreo usando la menor cantidad de hardware posible. En tercer lugar, mantener el sistema de monitoreo sin el desarrollo de una base de datos ni HMI para almacenamiento y visualización de data, así como rutas autónomas. Finalmente, como restricciones se tienen los sistemas mecatrónicos y pruebas in situ, por lo que, en primera instancia, los resultados se obtendrán de un ambiente controlado. 