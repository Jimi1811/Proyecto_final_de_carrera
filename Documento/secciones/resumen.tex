\customchapter{RESUMEN}

La acuicultura es una industria emergente con mayor presencia en el mercado, siendo el salmón uno de los peces más populares. Existen diferentes métodos y estructuras de cultivo, siendo uno de los más utilizados los criaderos de peces en jaulas hechas de redes ubicadas en mar abierto. A medida que aumenta el volumen de producción, se requiere automatización en su mantenimiento. Uno de los principales desafíos es la detección de agujeros y suciedad, y el método común es la inspección de buzos marinos. Esta tesis presenta una investigación práctica para el monitoreo en tiempo real de los criaderos de peces utilizando herramientas de visión por computadora e inteligencia artificial. El objetivo principal es desarrollar un sistema capaz de identificar y cuantificar la integridad de la red y el nivel de suciedad en las jaulas. En primer lugar, para los agujeros en la red, se desarrolla un método de aprendizaje profundo utilizando un algoritmo modificado de Mask R-CNN para lograr una detección robusta y mejorada. En segundo lugar, para la presencia de suciedad, se utilizan diferentes filtros de procesamiento de imágenes para obtener un área estimada creada por esta. Además, para una visualización completa de la red de peces, se crea una reconstrucción utilizando monocular SLAM visual inercial de presión. Los resultados preliminares están relacionados con el algoritmo de aprendizaje profundo, donde se crea un conjunto de datos de redes de peces utilizando diferentes imágenes de una búsqueda en la web.


\noindent \textbf{Palabras clave:}\\
\noindent  Visual inertial pressure monocular SLAM; Robotics; Deep Learning; Computer vision; Aquaculture